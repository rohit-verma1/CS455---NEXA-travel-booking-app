\subsection{External Interface Requirements}

\subsubsection{User Interfaces}
The Nexa system will be delivered as a responsive web application accessible on desktops, laptops, tablets, and smartphones. 
The frontend will be built using EXT.js, TailwindCSS, providing a clean layout and adaptive design across all screen sizes. 

For \textbf{customers}, the interface will include a search bar for entering route, date, and travel type. 
Search results will support filtering (price, amenities, rating, departure time) and sorting (price, duration, rating). 
Real-time seat availability will be displayed, with interactive seat maps available where providers define layouts. 
The customer dashboard will include Profile, My Bookings, Payment History, Loyalty Points, Notifications, and Community Forums. 
Customers will also access travel blogs, FAQs, and AI-based itinerary planning.

For \textbf{providers}, the interface will feature a dashboard for managing services, schedules, pricing, and policies. 
It will also present bookings, occupancy rates, and revenue trends. 
Providers can respond to reviews, flag issues, and broadcast updates to passengers.

For \textbf{administrators}, the interface will provide system-wide control tools. 
Admins will be able to verify providers, moderate content, resolve disputes, and oversee flagged reports. 
Analytics dashboards will present system health (active users, bookings per minute), financial summaries, and usage trends. 
Administrators will also configure platform policies and issue announcements or maintenance notices.

All interfaces will follow accessibility guidelines (WCAG 2.1) and support usability features such as keyboard navigation, high-contrast mode, and optional multilingual support.

\subsubsection{Software Interfaces}
The backend will be implemented using Django in a microservices style. RESTful APIs (JSON) will connect frontend and backend, as well as enable external integrations. Redis will be used for caching and lightweight queues.

\textbf{Operating Environment:}
\begin{itemize}
    \item Backend services and the primary PostgreSQL database will run on a Linux-based cloud or local server.
    \item The client-side web app will support all modern browsers (Chrome, Safari, Firefox, Edge, Brave).
\end{itemize}

\textbf{Databases (/Domains) :}
PostgreSQL will serve as the relational database. Related features will be grouped into logical tables with JSONB fields for flexibility. The main domains include:
\begin{itemize}
    \item \textbf{Users:} Stores all user accounts information (customers, providers, and administrators) along with authentication and role information.
    \item \textbf{Services \& Schedules:} Stores travel services (bus, train, flight) and their associated schedules, routes, live location, and basic service details.
    \item \textbf{Bookings:} Stores booking records, passenger details, payment status (mock), cancellations, and reschedules.
    \item \textbf{Reviews \& Reports:} Stores customer ratings, feedback, and issue reports related to services or providers.
    \item \textbf{Forums:} Stores discussion threads, replies, and votes for community interactions.
    \item \textbf{Notifications:} Stores alerts, confirmations, and promotional messages sent to users.
    \item \textbf{Loyalty \& Referrals:} Stores loyalty point transactions and referral program records.
    \item \textbf{AI Features:} Stores previous customer interactions, generated itineraries, and chat history to support AI-driven features.
\end{itemize}


\textbf{External Services:}
\begin{itemize}
    \item \textbf{Payment Simulation:} A mock module will simulate card/UPI transactions to test booking workflows without real payments.
    \item \textbf{Notification Simulation:} Email/SMS alerts will be tested using console outputs, local mail servers, or free-tier services (e.g., Mailtrap, Gmail API).
    \item \textbf{Tracking APIs:} Free-tier or open APIs (e.g., FlightRadar, Indian Railways) or mock GPS feeds will provide live tracking data.
    \item \textbf{Authentication:} Social login via OAuth 2.0 may be enabled using free-tier API keys (Google, GitHub).
    \item \textbf{AI/ML Modules:} Itinerary planning, demand forecasting, and recommendation services will be provided by internal microservices. 
    Where necessary, free-tier LLM APIs (e.g., Google Gemini or OpenAI free credits) may be leveraged for chatbot-style trip planning and contextual guidance.
\end{itemize}
